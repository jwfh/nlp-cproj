\documentclass{article}

\parindent=0pt
\parskip=1pc
\textwidth=5.25in
\textheight=8.0in
\topmargin=-0.1in
\oddsidemargin=0.7in

\begin{document}
	{\bfseries
		Computer Science 4750 \hfill Fall 2018 \\
		Course Project Proposal Notes \hfill Monday, September 17%
	}
	\begin{center}
		\Large\bfseries Natural Language Processing
	\end{center}

	Over the next 6 weeks, you should choose the topic of your course project. This project can take the form of a literature survey on an NLP topic of interest to you or a software system implementing a particular NLP task. {\em \bfseries Each course project MUST be approved by the course instructor}; please do this by chatting with your course instructor by October 31st. Course projects that are not approved prior to the submission of the associated project proposal (see below) will receive a mark of zero.
	
	Once a course project is approved, you must write and submit a project proposal due at noon on Thursday, November 1, and worth 5 course marks. This proposal should be about a page long and consist of a 3-paragraph proposal text (3.5 marks total) (with the first paragraph motivating why your topic is of interest (1 mark), the second summarizing previous work on this topic (1.5 marks), and the third summarizing the focus of your survey or the approach you will use in constructing your software system (1 mark)) followed by at least five full-information literature references particular to the topic of your project which must each be cited at least once in your proposal text (1.5 marks total) (0.3 marks per reference up to 5 references).
	
	Once the project proposal is submitted, you have until midnight on Friday, November 30, to do your project. In general, In general, this project will entail (as a minimum) a 15-20 page report (double-spaced) with 10-30 literature references (for a literature-survey project) or a 5-8 page report (double-spaced) with 5-15 references in addition to the software (for a software project). {\em \bfseries All included references MUST be cited in the body of the report.} Though purely Web-based references (e.g., blogs, reference manuals) are acceptable, please try where at all possible to obtain literature-based references, e.g., books, book chapters, journal papers, conference papers; in those cases, {\em \bfseries You MUST give full references listing all reference information appropriate to the type of reference,} author names, publication year, full paper title, book title with editors, journal name, journal volume and number, publisher, page numbers.
	
	The length and reference-number requirements above may vary depending on the nature of the chosen project and whether or not this project is being done by one or more people; this should be settled with the course instructor. Ideally, the submitted project should be on the same topic as that described in your project proposal. However, it being an imperfect world, if any difficulties do arise, chat with your course instructor as soon as possible so appropriate action, e.g., revision of stated goals and/or scope of project, can be taken.
	
	Each project will also have an associated short in-class presentation scheduled in late November and early December. Details of talk format and scheduling will be posted in early November after project proposals are submitted.
	
	Here's to each of you choosing and carrying out a fun course project!
\end{document}